\documentclass[11pt]{beamer}

% This file is a solution template for:

% - Talk at a conference/colloquium.
% - Talk length is about 20min.
% - Style is ornate.

% Copyright 2004 by Till Tantau <tantau@users.sourceforge.net>.
%
% In principle, this file can be redistributed and/or modified under
% the terms of the GNU Public License, version 2.
%
% However, this file is supposed to be a template to be modified
% for your own needs. For this reason, if you use this file as a
% template and not specifically distribute it as part of a another
% package/program, I grant the extra permission to freely copy and
% modify this file as you see fit and even to delete this copyright
% notice.


\mode<presentation>
{
  \usetheme{Warsaw}
  % or ...

  % \setbeamercovered{transparent}
  % or whatever (possibly just delete it)
}

\usepackage[icelandic]{babel}
% or whatever
\selectlanguage{icelandic}

\usepackage[latin1]{inputenc}
% or whatever

\usepackage{times}
\usepackage[T1]{fontenc}
\usepackage{float}
\usepackage{amssymb}
\usepackage[normalem]{ulem}
\usepackage{booktabs}
\usepackage{multirow}
\usepackage{xcolor}
\usepackage{listings}

\newcommand{\perm}{{\mathfrak S}}
\newcommand{\clpreim}{\texttt{ClassicalPreim}}
\newcommand{\meshpreim}{\texttt{MeshPreim}}
\renewcommand{\epsilon}{\varepsilon}

\DeclareMathOperator{\Av}{Av}

\lstset{ language=C++,
    basicstyle = \small \ttfamily,
    keywordstyle = \color{blue},
    stringstyle = \color{red},
    commentstyle = \color{green},
    showstringspaces = false,
    tabsize = 3,
    frame = single
}


\title{Symbol Tables and hashing}
\subtitle{Gagnaskipan 2014}

\author[Hjalti]{Hjalti Magn�sson (\href{mailto:hjaltim@ru.is}{hjaltim@ru.is})}

\institute[Reykjav�k University] % (optional, but mostly needed)
{
  \includegraphics[height=2cm]{rulogo}
}

\date{\today}

\AtBeginSection[]
{
  \begin{frame}<beamer>{Outline}
    \tableofcontents[currentsection,currentsubsection]
  \end{frame}
}

\AtBeginSubsection[]
{
  \begin{frame}<beamer>{Yfirlit}
    \tableofcontents[currentsection,currentsubsection]
  \end{frame}
}

\DeclareMathOperator{\id}{id}

\begin{document}

\begin{frame}
  \titlepage
\end{frame}

\begin{frame}{Searching}
    \begin{itemize}
        \item Important on many levels of CS
        \item Search by keys
            \begin{itemize}
                \item Dictionary
                \item Phone book
                \item Search engine
                \item Database
            \end{itemize}
    \end{itemize}
\end{frame}

\begin{frame}{Symbol Table}
    \begin{itemize}
        \item Aka \emph{map}, \emph{dictionary}, \emph{associative array}
        \item A set of $(\mathrm{key}, \mathrm{value})$ pairs
        \item Operations
            \begin{itemize}
                \item Add a new pair
                \item Return an item with a given key
            \end{itemize}
        \item Alternatively: An array (or vector) that can be indexed by anything
%         \item Alternatively: Mathematical concept of map
%             \begin{itemize}
%                 \item Possible keys form the domain of the map
%                 \item Values form the image (codomain) of the map
%             \end{itemize}
    \end{itemize}
\end{frame}

\begin{frame}{Implementations}
    \begin{itemize}
        \item Array indexed by key
        \item Vector of pairs
        \item List of pairs
        \item Sorted vector of pairs
        \item Binary Search Tree
    \end{itemize}
\end{frame}

\begin{frame}{Implementations of symbol table}
    \begin{table}
        \centering
        \begin{tabular}{lccc}
            \textbf{Method}     & \textbf{Get}     & \textbf{Add} & \textbf{Remove} \\
            \color<2>{red}{Key-indexed array}   & \color<2>{red}{$O(1)$}                & \color<2>{red}{$O(1)$}         & \color<2>{red}{$O(1)$} \\
            Unordered vector    & $O(n)$                & $O(1)$          & $O(n)$ \\
            Unordered list      & $O(n)$                & $O(1)$          & $O(n)$ \\
            Ordered list        & $O(n)$                & $O(n)$          & $O(n)$ \\
            Binary search       & $O(\log(n))$          & $O(n)$          & $O(n)$ \\
            BST (Average)       & $O(\log(n))$          & $O(\log(n))$    & $O(\log(n))$ \\
            BST (WC)            & $O(n)$                & $O(n)$          & $O(n)$
        \end{tabular}
    \end{table}
\end{frame}

\begin{frame}{Hashing -- The major players}
    \begin{itemize}
        \item The \emph{universe} of elements, $U$
            \begin{itemize}
                \item The set of all integers ($2^{32}$)
                \item The set of all floating point numbers ($\sim 2^{64}$)
                \item Points in the integer lattice ($2^{64}$)
                \item The set of all strings
                    \begin{itemize}
                        \item More strings of length 40 than there are atoms in the universe
                    \end{itemize}
            \end{itemize}
        \item The \emph{hash function}, $h$
            \begin{itemize}
                \item Maps elements of $U$ to $M$ \emph{buckets}
            \end{itemize}
        \item The \emph{hash map} $H$ is an array of element from $U$ (indexed by integers)
            \begin{itemize}
                \item If $u \in U$, $h(u)$ gives the index of the element $u$ in $H$
            \end{itemize}
    \end{itemize}
\end{frame}

\begin{frame}{Hashing}
    \begin{block}{Hash function}
        A hash function $h: U \to [0,M-1]$, is function that maps data of
        variable length to data of fixed length.
    \end{block}
    \begin{block}{Hash function}
        A function that turns a (long) sequence of bits into a fixed length sequence of bits.
    \end{block}
\end{frame}

\begin{frame}{What is a good hash function?}
    \begin{itemize}
        \item Simplicity
            \begin{itemize}
                \item Lines of code (number of instructions)
            \end{itemize}
        \item Speed
            \begin{itemize}
                \item CPU benchmarks
            \end{itemize}
        \item Strength
            \begin{itemize}
                \item Hard to measure
                \item Few \emph{collisions}
                \item Good distribution
                \item Avalanche condition
            \end{itemize}
        \item Security?
    \end{itemize}
\end{frame}

\begin{frame}{Hashing strings}
    \begin{itemize}
        \item All hashing can be reduced to hashing strings
        \item Reduce a sequence of bytes (8-bit blocks) to 32 or 64 bits (possibly more).
    \end{itemize}
\end{frame}

\begin{frame}[fragile]
    \frametitle{Idea 1}
    \begin{lstlisting}
int hash(const char *str, int size) {
    return size;
}
    \end{lstlisting}
    \begin{itemize}
        \item<2-> Not good!
        \item<3-> Seriously, not good!
        \item<4-> DO NOT USE!
        \item<5> Once used by PHP
    \end{itemize}
\end{frame}

\begin{frame}[fragile]
    \frametitle{Idea 2 -- LoseLose}
    \begin{lstlisting}
int hash(const char *str, int size) {
    int hash = 0;
    for(int i = 0; i < size; i++) {
        hash += str[i];
    }
    return hash;
}
    \end{lstlisting}
    \begin{itemize}
        \item<2-> A little bit better
        \item<3-> Still terrible
        \item<4> Appeared in \emph{The C Programming Language}
    \end{itemize}
\end{frame}

\begin{frame}[fragile]
    \frametitle{Idea 3 -- djb2}
    \begin{lstlisting}
int hash(const char *str, int size) {
    int hash = 5381;
    for(int i = 0; i < size; i++) {
        hash = hash * 33 + s[i];
    }
    return hash;
}
    \end{lstlisting}
    $$
        h(w_n \dotsc w_1) = 5381 \cdot 33^n + w_1 \cdot 33^{n-1} + w_2 \cdot 33^{n-2} + \dotsb + w_n
    $$
    \begin{itemize}
        \item<2-> Created by Daniel J. Bernstein
        \item<3-> Efficient
        \item<4-> Used by Java and C\# until recently
        \item<5> (Susceptible to attacks)
    \end{itemize}
\end{frame}

\begin{frame}{Modern hashes}
    \begin{itemize}
        \item MurmurHash
            \begin{itemize}
                \item Currently used by Java
            \end{itemize}
        \item lookup3
        \item One-at-a-Time Hash
        \item FNV
        \item SuperFastHash
    \end{itemize}
\end{frame}

\begin{frame}[fragile]
    \frametitle{MurmurHash}
    \begin{lstlisting}
int MurmurHash2(char *str, int len, int seed) {
    int m = 0x5bd1e995;
    int r = 24; int h = seed ^ len;
    while(len >= 4) { int k = *str;
        k *= m; k ^= k >> r; k *= m;
        h *= m; h ^= k; str += 4; len -= 4;
    }
    switch(len) {
        case 3: h ^= data[2] << 16;
        case 2: h ^= data[1] << 8;
        case 1: h ^= data[0];
                h *= m;
    }
    h ^= h >> 13; h *= m; h ^= h >> 15;
    return h;
}
    \end{lstlisting}
\end{frame}

\begin{frame}{Hash table}
    \begin{itemize}
        \item Create an array, $H$, of size $M$
        \item $H$ is called a \emph{hash map}
        \item Store $u \in U$ in position $h(u)$ in $A$
    \end{itemize}
\end{frame}

\begin{frame}{Collisions}
    \begin{itemize}
        \item A \emph{collision} is when
            \begin{itemize}
                \item $u,v \in U$,
                \item $u \neq v$, and
                \item $h(u) = h(v)$
            \end{itemize}
        \item Resolving collisions
            \begin{itemize}
                \item Open addressing
                \item Chaining
                \item Cuckoo hashing
            \end{itemize}
    \end{itemize}
\end{frame}

\end{document}

